%*******************************************************************************************%
%	   CONTRIBUIÇÕES À MODELAGEM DINAMICA E AO CONTROLE DE MANIPULADORES PARALELOS          %
% 																					   		%
% June 2015															 						%
% Author: Renato Maia Matarazzo Orsino														%
% bash modularmodelling.sh					 												%
% 																							%
%*******************************************************************************************%


\documentclass[25pt,landscape]{beamer}
	\mode<presentation> {
	  \usetheme{Warsaw}
	  \setbeamercovered{transparent}
	  \useoutertheme{shadow}
	  \useoutertheme{infolines}
	  \useinnertheme{rounded}
	  \setbeamertemplate{theorems}[numbered]
	  \setbeamertemplate{bibliography item}[text]
	  \usecolortheme{default}
	}


% - INPUT - %
\usepackage[T1]{fontenc}
\usepackage[latin1]{inputenc}
\usepackage[brazil]{babel}


% - MATH - %
\usepackage{amsmath,amsfonts,amssymb}
\usepackage{amsthm}
\usepackage{accents}


% - GENERAL - %
\usepackage{cmbright}
\usepackage{natbib}
\usepackage{color}
% \usepackage[pdftex,dvipsnames]{color}
% \usepackage{geometry}
\usepackage{array,hhline,supertabular,tabularx}
\usepackage{hyperref}
% \usepackage[colorlinks,citecolor=black,urlcolor=black,linkcolor=black]{hyperref}
\usepackage{graphicx}
% \usepackage[pdftex]{graphicx}
\usepackage{enumitem}
\usepackage{float}
\usepackage{titlesec}
\usepackage{pp4link}
\usepackage{mpmulti}
\usepackage{multimedia}
\usepackage[display]{texpower}
\usepackage{multicol}
\usepackage[overlay,absolute]{textpos}
\setlength{\TPHorizModule}{1cm}
\setlength{\TPVertModule}{1cm}
\usepackage{multicol}

% - SPECIAL - %
\usepackage{EXTRAS/special-char}
\usepackage{EXTRAS/special-conf}




%---------------------------------------------------------------------------------------%
%										BEAMER											%
%---------------------------------------------------------------------------------------%

\begin{document}


\begin{frame}
    \titlepage
\end{frame}



%============================================== REVISÃO DA LITERATURA =======================================================%


\begin{frame}{Revisão da Literatura}
    \framesubtitle{Modelagem Dinâmica}
    \pause
        \begin{block}{Influência da topologia do robô}
    	\begin{itemize}
    		\pause
    		\item[$\bullet$] Seriais
    		\begin{itemize}
    			\pause
    			\item[--] Cadeia aberta \\[4pt]
    			\item[--] Juntas ativas de 1 gl \\[4pt]
    			\item[--] N$^\circ$ de coord. gen. = N$^\circ$ atuadores = mobilidade \\[4pt]
    			\item[--] Conjunto mínimo de coord. generalizadas \\[4pt]
    			\item[--] Cinemática direta simples \\[4pt]
    			\item[--] Cinemática inversa complexa \\[4pt]
    			\item[--] Dinâmica direta - Sistema de EDOs \\[4pt]
    			\item[--] Dinâmica inversa - Sistema linear \\[4pt]
    			\item[--] Algoritmos recursivos para mod. dinâmica \\[4pt]
    		\end{itemize}
    	\end{itemize}
    \end{block}
    $$ $$
\end{frame}

\begin{frame}{Revisão da Literatura}
    \framesubtitle{Modelagem Dinâmica}
    \begin{block}{Influência da topologia do robô}
    	\begin{itemize}
    		\item[$\bullet$] Paralelos
    		\begin{itemize}
    			\pause
    			\item[--] Cadeia fechada \\[4pt]
    			\item[--] Juntas de 1, 2 ou 3 gl, ativas ou passivas \\[4pt]
    			\item[--] Grande número de elos \\[4pt]
    			\item[--] Grande quantidade de variáveis cinemáticas \\[4pt]
    			\item[--] Variáveis independentes e dependentes \\[4pt]
    			\item[--] Cinemática direta complexa \\[4pt]
    			\item[--] Cinemática inversa "simples" \\[4pt]
    			\item[--] Dinâmica direta - Sistema de EDAs ou EDOs \\[4pt]
    			\item[--] Dinâmica inversa - Sistema não linear \\[4pt]
    			\item[--] Coord. gen. ind.: coord. dos atuadores ou do efetuador \\[4pt]
    		\end{itemize}
    	\end{itemize}
    \end{block}
\end{frame}

\begin{frame}{Revisão da Literatura}
    \framesubtitle{Modelagem Dinâmica}
    \begin{block}{Dinâmica direta - EDAs}
		\begin{equation} \label{DAE1} \tag{2.1}
			\mM \, \ddot{\mq} + \mA^\msT \,\mlambda = \meta
		\end{equation}
		\begin{equation} \label{DAE2} \tag{2.2}
			\bar{\mq} \, (\mq,t) \, = \, \mzr
		\end{equation}
	
		Sendo
		\begin{equation} \tag{2.3}
			\mA(\mq,t) = \frac{\partial \bar{\mq}}{\partial \mq}
		\end{equation}
	\end{block}
\end{frame}

\begin{frame}{Revisão da Literatura}
    \framesubtitle{Modelagem Dinâmica}	
	\begin{block}{Dinâmica direta - EDOs}	
		\begin{equation} \label{DAE3} \tag{2.4}
			\underbrace{\left[ \begin{array}{cc}
			\mM & \mA^\msT \\
			\mA & \mzr
			\end{array}
			\right]}_{\mY}
			\left[ \begin{array}{c}
			\ddot{\mq} \\
			\mlambda
			\end{array}
			\right] =
			\left[ \begin{array}{c}
			\meta \\
			-\mb
			\end{array}
			\right]
		\end{equation}
		Sendo
	\begin{equation} \tag{2.5}
		\mb = \frac{\partial (\mA \dot{\mq})}{\partial \mq} \, \dot{\mq} + 2 \frac{\partial \mA}{\partial t} \, \dot{\mq} + \frac{\partial^2 \bar{\mq}}{\partial t^2}
	\end{equation}
	\end{block}
	\pause
	\begin{block}{Método estabilização de Baumgarte}
		\begin{equation} \label{baumgarte} \tag{2.6}
			\mb' = \mb + 2\hat{\alpha} \dot{\bar{\mq}} + \hat{\beta}^2 \bar{\mq}
		\end{equation}
	\end{block}	
\end{frame}

\begin{frame}{Revisão da Literatura}
    \framesubtitle{Modelagem Dinâmica}
        \begin{block}{Propósito}
    	\begin{itemize}
    		\pause
    		\item[$\bullet$] Simulação
    		\begin{itemize}
    			\pause
    			\item[--] Projeto/Dimensionamento do mecanismo/manipulador \\[4pt]
    			\item[--] Grau de detalhamento do modelo depende da aplicação \\[4pt]
    			\item[--] Não necessita rodar em tempo real \\[4pt]
    		\end{itemize}
    		\pause
    		\item[$\bullet$] Controle
    		\begin{itemize}
    			\pause
    			\item[--] Projeto do controlador \\[4pt]
    			\item[--] Compensação de não linearidades \\[4pt]
    			\item[--] Modelos demasiadamente complexos dificultam o projeto e podem aumentar o custo computacional \\[4pt]
    			\item[--] Modelos muito simplistas podem comprometer o desempenho \\[4pt]
    			\item[--] Muitas vezes precisa rodar em tempo real \\[4pt]
    		\end{itemize}
    	\end{itemize}
    \end{block}
\end{frame}

\begin{frame}{Revisão da Literatura}
    \framesubtitle{Modelagem Dinâmica}
    \begin{block}{Principais formulações}
        \begin{itemize}
            \item[$\bullet$] Formalismo de Newton-Euler (Arian \emph{et al.}, 2017; Zhang \emph{et al.}, 2014) \\[4pt] % Dasgupta e Mruthyunjaya, 1998; Li \emph{et al.}, 2003; Shiau \emph{et al.}, 2008;
            \item[$\bullet$] Formalismo de Lagrange (Singh e Santhakumar, 2015; Yao \emph{et al.}, 2017) \\[4pt] %Li e Xu, 2005; Singh \emph{et al.}, 2015; Singh \emph{et al.}, 2014;
            \item[$\bullet$] Princípio dos Trabalhos/Potências Virtuais (Gallardo-Alvarado \emph{et al.}, 2018; Li e Staicu, 2012) \\[4pt] %Codourey, 1996; Codourey e Burdet, 1997; Geike e McPhee, 2003; Li e Xu, 2009; Staicu, 2009; Staicu e Carp-Ciocardia, 2003; Staicu \emph{et al.}, 2007; Staicu e Zhang, 2008; Staicu \emph{et al.}, 2006; Wu \emph{et al.}, 2009; Zhao e Gao, 2009; Zhu \emph{et al.}, 2005;
            \item[$\bullet$] Formulação Lagrange-D'Alambert (Cheng \emph{et al.}, 2001; Yen e Lai, 2009) \\[4pt]
            \item[$\bullet$] Método de Kane (Ben-Horina \emph{et al.}, 1998; Shukla e Karki, 2014) \\[4pt]
            \item[$\bullet$] Formalismo de Boltzmann-Hammel (Abdellatif e Heimann, 2009; Altuzarra \emph{et al.}, 2015) \\[4pt]
            \item[$\bullet$] Formulação do Complemento Ortogonal Natural  (Akbarzadeh \emph{et al.}, 2013; Khan \emph{et al.}, 2005) \\[4pt] %Xi e Sinatra, 2002; 
        \end{itemize}
    \end{block}
\end{frame}

\begin{frame}{Revisão da Literatura}
    \framesubtitle{Controle}
    \pause
    \begin{block}{Principais técnicas}
        \begin{itemize}
            \item[$\bullet$] Controle Proporcional-Integral-Derivativo \\
            \item[$\bullet$] Controle por Torque Computado (Shang e Cong, 2009; Yen e Lai, 2009) \\ % Cheng \emph{et al.}, 2003; Li e Wu, 2004; Li e Xu, 2009;  (22; 55; 56; 75; 106)
            \item[$\bullet$] Controle por Torque Computado com pré-alimentação (Siciliano \emph{et al.}, 2010; Spong \emph{et al.}, 2006)\\  % Khalil e Dombre, 2002;  (48; 78; 85)
            \item[$\bullet$] Controle por Torque Computado Estendido (Zubizarreta \emph{et al.}, 2013; Zubizarreta \emph{et al.}, 2012) \\ % Zubizarreta \emph{et al.}, 2010; Zubizarreta \emph{et al.}, 2008;(113; 114; 115; 116)
            \item[$\bullet$] Controle Preditivo Baseado em Modelo (Duchaine \emph{et al.}, 2007; Vivas e Poignet, 2005) \\ %(32, 100)
            \item[$\bullet$] Controle Adaptativo (Chemori \emph{et al.}, 2013; Honegger \emph{et al.}, 2000)  \\ %Codourey e Burdet, 1997; Slotine e Li, 1987; (19; 25; 41; 84)
            \item[$\bullet$] Controle por Modos Deslizantes (Hu e Woo, 2006; Sadati e Ghadami, 2008)  \\ % Begon \emph{et al.}, 1995; Ertugrul e Kaynak, 2000; (9; 34; 42; 73)
        \end{itemize}
    \end{block}
\end{frame}

\begin{frame}{Revisão da Literatura}
    \framesubtitle{Controle}
    \begin{block}{Controle Proporcional-Integral-Derivativo (PID)}
        \begin{itemize}
            \item[$\bullet$] Técnica de controle linear descentralizado \\[8pt]
            \item[$\bullet$] Não baseado no modelo dinâmico do mecanismo \\[8pt]
            \item[$\bullet$] Simples implementação \\[8pt]
            \item[$\bullet$] Baixo custo computacional \\[8pt]
            \item[$\bullet$] Desempenho bastante limitado \\[8pt]
        \end{itemize}
    \end{block}
\end{frame}

\begin{frame}{Revisão da Literatura}
    \framesubtitle{Controle}
    \begin{block}{Controle por Torque Computado (CTC)}
        \begin{itemize}
            \item[$\bullet$] Técnica de controle não linear centralizado \\[8pt]
            \item[$\bullet$] Baseado no modelo dinâmico do mecanismo \\[8pt]
            \item[$\bullet$] Realiza compensação de não linearidades por realimentação \\[8pt]
            \item[$\bullet$] Desempenho superior ao PID, porém bastante dependente da qualidade do modelo dinâmico \\[8pt]
            \item[$\bullet$] Implementação mais complexa \\[8pt]
            \item[$\bullet$] Maior custo computacional \\[8pt]
        \end{itemize}
    \end{block}
\end{frame}

\begin{frame}{Revisão da Literatura}
    \framesubtitle{Controle}
    \begin{block}{Controle por Torque Computado com pré-alimentação (CTCp)}
        \begin{itemize}
            \item[$\bullet$] Lei de controle similar ao CTC \\[8pt]
            \item[$\bullet$] Realiza compensação de não linearidades por pré-alimentação \\[8pt]
            \item[$\bullet$] Menor custo computacional em relação ao CTC \\[8pt]
            \item[$\bullet$] Implementação mais simples que o CTC \\[8pt]
            \item[$\bullet$] Menor robustez em relação ao CTC \\[8pt]
        \end{itemize}
    \end{block}
\end{frame}

\begin{frame}{Revisão da Literatura}
    \framesubtitle{Controle}
    \begin{block}{Controle por Torque Computado Estendido (CTCe)}
        \begin{itemize}
            \item[$\bullet$] Lei de controle similar ao CTC \\[8pt]
            \item[$\bullet$] Realiza compensação de não linearidades por realimentação \\[8pt]
            \item[$\bullet$] Utiliza informação redundante obtida pelo sensoriamento de juntas passivas na lei de controle p/ aumentar robustez \\[8pt]
            \item[$\bullet$] Robusto a incertezas nos parâmetros cinemáticos \\[8pt]
        \end{itemize}
    \end{block}
\end{frame}

\begin{frame}{Revisão da Literatura}
    \framesubtitle{Controle}
    \begin{block}{Controle Preditivo Baseado em Modelo (CPM)}
        \begin{itemize}
            \item[$\bullet$] Técnica de controle multi-variável baseado em modelo \\[8pt]
            \item[$\bullet$] Muito utilizado no controle de processos industriais \\[8pt]
            \item[$\bullet$] Realiza otimização em tempo real de uma custo que envolve o erro e o esforço de controle em tempo futuro \\[8pt]
            \item[$\bullet$] Custo computacional bastante dependente da complexidade do modelo \\[8pt]
            \item[$\bullet$] Maior robustez a incertezas paramétricas  \\[8pt]
        \end{itemize}
    \end{block}
\end{frame}

\begin{frame}{Revisão da Literatura}
    \framesubtitle{Controle}
    \begin{block}{Controle Adaptativo (CA)}
        \begin{itemize}
            \item[$\bullet$] Técnica de controle baseado em modelo \\[8pt]
            \item[$\bullet$] Estimação em tempo real de parâmetros do sistema \\[8pt]
            \item[$\bullet$] Baixa sensibilidade a incertezas paramétricas \\[8pt]
            \item[$\bullet$] Necessita de modelo dinâmico linear em relação aos parâmetros \\[8pt]
            \item[$\bullet$] Alternativamente pode realizar a estimação termos não lineares de compensação dinâmica \\[8pt]
            \item[$\bullet$] Custo computacional adicional relativo a integração das leis de adaptação \\[8pt]
            \item[$\bullet$] Maior complexidade de projeto e implementação \\[8pt]
        \end{itemize}
    \end{block}
\end{frame}

\begin{frame}{Revisão da Literatura}
    \framesubtitle{Controle}
    \begin{block}{Controle por Modos Deslizantes (CMD)}
        \begin{itemize}
            \item[$\bullet$] Técnica de controle não linear robusto \\[8pt]
            \item[$\bullet$] Alta robustez em relação a incertezas estruturadas e não estuturadas \\[8pt]
            \item[$\bullet$] Desempenho menos dependente da qualidade do modelo dinâmico \\[8pt]
            \item[$\bullet$] Utiliza funções descontínuas na lei de controle, o que pode causar \emph{chattering} \\[8pt]
        \end{itemize}
    \end{block}
\end{frame}









%-----------------------------------------------------------------------------------------------------------------------------------

\end{document}